%===================================================================================
% JORNADA CIENTÍFICA ESTUDIANTIL - MATCOM, UH
%===================================================================================
% Esta plantilla ha sido diseñada para ser usada en los artículos de la
% Jornada Científica Estudiantil, MatCom.
%
% Por favor, siga las instrucciones de esta plantilla y rellene en las secciones
% correspondientes.
%
% NOTA: Necesitará el archivo 'jcematcom.sty' en la misma carpeta donde esté este
%       archivo para poder utilizar esta plantila.
%===================================================================================



%===================================================================================
% PREÁMBULO
%-----------------------------------------------------------------------------------
\documentclass[a4paper,10pt,twocolumn]{article}

%===================================================================================
% Paquetes
%-----------------------------------------------------------------------------------
\usepackage{amsmath}
\usepackage{amsfonts}
\usepackage{amssymb}
\usepackage{jcematcom}
\usepackage[utf8]{inputenc}
\usepackage{listings}
\usepackage[pdftex]{hyperref}
%-----------------------------------------------------------------------------------
% Configuración
%-----------------------------------------------------------------------------------
\hypersetup{colorlinks,%
	    citecolor=black,%
	    filecolor=black,%
	    linkcolor=black,%
	    urlcolor=blue}

%===================================================================================



%===================================================================================
% Presentacion
%-----------------------------------------------------------------------------------
% Título
%-----------------------------------------------------------------------------------
\title{Informe del Primer Mini Proyecto de Estad\'istica}

%-----------------------------------------------------------------------------------
% Autores
%-----------------------------------------------------------------------------------
\author{\\
	\name Roberto Marti Cede\~no \email \href{mailto:r.marti@estudiantes.matcom.uh.cu}{r.marti@estudiantes.matcom.uh.cu}
	\\ \addr Grupo C412 \AND
	\name Daniel Alberto Garc\'ia P\'erez \email \href{mailto:d.garcia@estudiantes.matcom.uh.cu}{d.garcia@estudiantes.matcom.uh.cu}
	\\ \addr Grupo C412 \AND
	\name Leonel Alejandro Garc\'ia L\'opez \email \href{mailto:l.garcia3@estudiantes.matcom.uh.cu}{l.garcia3@estudiantes.matcom.uh.cu}
	\\ \addr Grupo C412}

%-----------------------------------------------------------------------------------
% Tutores
%-----------------------------------------------------------------------------------
\tutors{\\Msc. Dalia Diaz Sistachs, \emph{Facultad de Matemática y Computación, Universidad de La Habana} \\}

%-----------------------------------------------------------------------------------
% Headings
%-----------------------------------------------------------------------------------
\jcematcomheading{\the\year}{1-\pageref{end}}{A. Uno, A. Dos}

%-----------------------------------------------------------------------------------
\ShortHeadings{Primer Mini Proyecto de Estad\'istica}{Roberto Marti Cede\~no , Daniel Alberto Garc\'ia P\'erez, Leonel Alejandro Garc\'ia L\'opez}
%===================================================================================



%===================================================================================
% DOCUMENTO
%-----------------------------------------------------------------------------------
\begin{document}

%-----------------------------------------------------------------------------------
% NO BORRAR ESTA LINEA!
%-----------------------------------------------------------------------------------
\twocolumn[
%-----------------------------------------------------------------------------------

\maketitle

%===================================================================================
% Resumen y Abstract
%-----------------------------------------------------------------------------------
\selectlanguage{spanish} % Para producir el documento en Español

%-----------------------------------------------------------------------------------
% Resumen en Español
%-----------------------------------------------------------------------------------


%-----------------------------------------------------------------------------------
% Palabras clave
%-----------------------------------------------------------------------------------
%\begin{keywords}
%	Separadas,
%	Por,
%	Comas.
%\end{keywords}

%-----------------------------------------------------------------------------------
% Temas
%-----------------------------------------------------------------------------------
%\begin{topics}
%	Tema, Subtema.
%\end{topics}


%-----------------------------------------------------------------------------------
% NO BORRAR ESTAS LINEAS!
%-----------------------------------------------------------------------------------
\vspace{0.8cm}
]
%-----------------------------------------------------------------------------------


%===================================================================================

%===================================================================================
% Introducción
%-----------------------------------------------------------------------------------
\section{Introducción}\label{sec:intro}
%-----------------------------------------------------------------------------------
  En esta sección puede incluir una presentación del dominio de su problema,
  los objetivos y motivaciones fundamentales de su investigación así como un
  resumen del estado del arte al respecto.

%===================================================================================



%===================================================================================
% Desarrollo
%-----------------------------------------------------------------------------------
\section{Ejercicios}\label{sec:dev}
%-----------------------------------------------------------------------------------
	Mostramos a continuación los enunciados de los ejercicios propuestos.

%-----------------------------------------------------------------------------------
	\subsection{Ejercicio 1}\label{sub:E1}
%-----------------------------------------------------------------------------------
	Ajuste los siguientes datos a una recta por mínimos cuadrados:
	
	\begin{table}[h]
		\centering
		\begin{tabular}{|c|c|c|c|c|}
			\hline
			x & -1 & 0 & 3 & 7\\ \hline
			F(x) & 2 & 0 & 4 & 7 \\ \hline
		\end{tabular}
		\label{table:EJ1}
	\end{table}

	\begin{enumerate}
		\item {Construya el diagrama de dispersión}
		\item {Estime el valor de la función para x = 1}
	\end{enumerate}

%-----------------------------------------------------------------------------------
	\subsection{Ejercicio 2}\label{sub:OEJ2}
%-----------------------------------------------------------------------------------
	Ajuste los siguientes datos a una recta por mínimos cuadrados:
	
	\begin{table}[h]
		\centering
		\begin{tabular}{|c|c|c|c|c|c|c|}
			\hline
			x & -3 & -1 & 1 & 3 & 5 & 7 \\ \hline
			F(x) & 14 & 4 & 2 & 8 & 22 & 44 \\ \hline
		\end{tabular}
		\label{table:EJ2}
	\end{table}

	\begin{enumerate}
		\item {Construya el diagrama de dispersión}
		\item {Estime el valor de la función para x = 0 y x = 2}
	\end{enumerate}

	\subsection{Ejercicio 3}
	La rapidez de pasos (número de pasos por segundo) es importante para el corredor serio. La rapidez de pasos está estrechamente relacionada con la velocidad y la meta del corredor es alcanzar la óptima rapidez de pasos. Como parte de un estudio, investigadores midieron la rapidez de pasos a 7 diferentes velocidades para 21 de las mejores corredoras del mundo, los valores del promedio de rapidez de pasos para estas mujeres y las velocidades de la prueba se indican en la tabla siguiente.
		
	\begin{table}[h]
		\centering
		\begin{tabular}{|c|c|}
			\hline
			Velocidad (ft/s) & F(x)  \\ \hline
			15.86 & 3.05 \\ \hline
			16.88 & 3.12 \\ \hline
			17.5  & 3.17 \\ \hline
			18.62 & 3.25 \\ \hline
			19.97 & 3.36 \\ \hline
			21.06 & 3.46 \\ \hline
			22.11 &	3.55 \\ \hline
		\end{tabular}
		\label{table:EJ3}
		
	
	\end{table}

	\begin{enumerate}
		\item {	Construya el diagrama de dispersión. ¿Parece ser lineal la relación? ¿Por qué?}
		\item {Encuentre la ecuación de la recta de mejor ajuste.}
		\item {Trace la recta en el diagrama de dispersión.}
		\item {Interprete la pendiente de la ecuación de regresión.}
		\item {Pronostique la rapidez de pasos promedio si la velocidad es de 19 pies por segundo}
		\item {Cuál es la rapidez de pasos si la velocidad es cero. Interprete sus resultados y Explique.}
	\end{enumerate}

%-----------------------------------------------------------------------------------
	\subsection{Figuras}\label{sub:figures}
%-----------------------------------------------------------------------------------
		Para producir cuerpos flotantes (figuras ó tablas), asegúrese de numerar
		y etiquetar correctamente cada figura. Las referencias a las figuras deben
		estar también correctamente etiquetadas. Por ejemplo, en la Fig. \ref{fig:ex}
		se muestra\ldots.

		\begin{figure}[htb]%
		\begin{center}
			\emph{Aquí va el contenido de la figura \ldots}
		\end{center}
		\caption{Figura de ejemplo \label{fig:ex}}%
		\end{figure}

%-----------------------------------------------------------------------------------
	\subsection{Código Fuente}\label{sub:listings}
%-----------------------------------------------------------------------------------
		Para producir código fuente, envuélvalo en una figura flotante y
		etiquételo correctamente. Por ejemplo, en la Fig. \ref{fig:code}
		se muestra un código bastante conocido\ldots.

		% Configuración de Listings
		\lstset{keywordstyle=\color{blue}, basicstyle=\small}

		\begin{figure}[htb]%
			\begin{lstlisting}[language=c]%

    int main(int argc, char** argv)
    {
        // Imprimiendo "Hola Mundo".
        printf("Hello, World");
    }

			\end{lstlisting}
		\caption{Código fuente de ejemplo.\label{fig:code}}
		\end{figure}

%-----------------------------------------------------------------------------------
	\subsection{Referencias}
%-----------------------------------------------------------------------------------
  	Las referencias deben estar agrupadas en una sección al final del artículo,
  	y las citas numeradas correctamente, por ejemplo \cite{knuth} ó \cite{goedel}.
  	Incluya toda la información importante de cada referencia, incluídos autor,
  	título, y notas de la edición. En caso de citar sitios web, además
  	de la URL, incluya la fecha en que fue consultado, como en \cite{wiki}.

%===================================================================================



%===================================================================================
% Conclusiones
%-----------------------------------------------------------------------------------
\section{Conclusiones}\label{sec:conc}

  En esta sección puede incluir las conclusiones de su investigación y las ideas
  sobre la continuidad del trabajo, en el caso que aplique.

%===================================================================================



%===================================================================================
% Recomendaciones
%-----------------------------------------------------------------------------------
\section{Recomendaciones}\label{sec:rec}

  En esta sección puede incluir recomendaciones sobre posibles formas de continuar
  la investigación u otros temas relacionados.

%===================================================================================



%===================================================================================
% Bibliografía
%-----------------------------------------------------------------------------------
\begin{thebibliography}{99}
%-----------------------------------------------------------------------------------
	\bibitem{knuth} Donald E. Knuth. \emph{The Art of Computer Programming}.
		Volume 1: Fundamental Algorithms (3rd~edition), 1997.
		Addison-Wesley Professional.

	\bibitem{goedel} Kurt Göedel. \emph{Über formal unentscheidbare Sätze der
		Principia Mathematica und verwandter Systeme, I}.
		Monatshefte für Mathematik und Physik 38.

	\bibitem{wiki} Wikipedia. URL: \href{http://en.wikipedia.org}
	  {http://en.wikipedia.org}.
		Consultado en \today.

%-----------------------------------------------------------------------------------
\end{thebibliography}

%-----------------------------------------------------------------------------------

\label{end}

\end{document}

%===================================================================================
