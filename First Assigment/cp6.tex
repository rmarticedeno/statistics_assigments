%===================================================================================
% JORNADA CIENTÍFICA ESTUDIANTIL - MATCOM, UH
%===================================================================================
% Esta plantilla ha sido diseñada para ser usada en los artículos de la
% Jornada Científica Estudiantil, MatCom.
%
% Por favor, siga las instrucciones de esta plantilla y rellene en las secciones
% correspondientes.
%
% NOTA: Necesitará el archivo 'jcematcom.sty' en la misma carpeta donde esté este
%       archivo para poder utilizar esta plantila.
%===================================================================================



%===================================================================================
% PREÁMBULO
%-----------------------------------------------------------------------------------
\documentclass[a4paper,10pt,twocolumn]{article}

%===================================================================================
% Paquetes
%-----------------------------------------------------------------------------------
\usepackage{amsmath}
\usepackage{amsfonts}
\usepackage{amssymb}
\usepackage{jcematcom}
\usepackage[utf8]{inputenc}
\usepackage{listings}
\usepackage[pdftex]{hyperref}
%-----------------------------------------------------------------------------------
% Configuración
%-----------------------------------------------------------------------------------
\hypersetup{colorlinks,%
	    citecolor=black,%
	    filecolor=black,%
	    linkcolor=black,%
	    urlcolor=blue}

%===================================================================================



%===================================================================================
% Presentacion
%-----------------------------------------------------------------------------------
% Título
%-----------------------------------------------------------------------------------
\title{Informe del Primer Mini Proyecto de Estad\'istica}

%-----------------------------------------------------------------------------------
% Autores
%-----------------------------------------------------------------------------------
\author{\\
	\name Roberto Marti Cede\~no \email \href{mailto:r.marti@estudiantes.matcom.uh.cu}{r.marti@estudiantes.matcom.uh.cu}
	\\ \addr Grupo C412 \AND
	\name Daniel Alberto Garc\'ia P\'erez \email \href{mailto:d.garcia@estudiantes.matcom.uh.cu}{d.garcia@estudiantes.matcom.uh.cu}
	\\ \addr Grupo C412 \AND
	\name Leonel Alejandro Garc\'ia L\'opez \email \href{mailto:l.garcia3@estudiantes.matcom.uh.cu}{l.garcia3@estudiantes.matcom.uh.cu}
	\\ \addr Grupo C412}

%-----------------------------------------------------------------------------------
% Tutores
%-----------------------------------------------------------------------------------
\tutors{\\Msc. Dalia Diaz Sistachs, \emph{Facultad de Matemática y Computación, Universidad de La Habana} \\}

%-----------------------------------------------------------------------------------
% Headings
%-----------------------------------------------------------------------------------
\jcematcomheading{\the\year}{1-\pageref{end}}{Roberto Marti Cede\~no , Daniel Alberto Garc\'ia P\'erez, Leonel Alejandro Garc\'ia L\'opez}

%-----------------------------------------------------------------------------------
\ShortHeadings{Primer Mini Proyecto de Estad\'istica}{Roberto Marti Cede\~no , Daniel Alberto Garc\'ia P\'erez, Leonel Alejandro Garc\'ia L\'opez}
%===================================================================================



%===================================================================================
% DOCUMENTO
%-----------------------------------------------------------------------------------
\begin{document}

%-----------------------------------------------------------------------------------
% NO BORRAR ESTA LINEA!
%-----------------------------------------------------------------------------------
\twocolumn[
%-----------------------------------------------------------------------------------

\maketitle

%===================================================================================
% Resumen y Abstract
%-----------------------------------------------------------------------------------
\selectlanguage{spanish} % Para producir el documento en Español

%-----------------------------------------------------------------------------------
% Resumen en Español
%-----------------------------------------------------------------------------------


%-----------------------------------------------------------------------------------
% Palabras clave
%-----------------------------------------------------------------------------------
%\begin{keywords}
%	Separadas,
%	Por,
%	Comas.
%\end{keywords}

%-----------------------------------------------------------------------------------
% Temas
%-----------------------------------------------------------------------------------
%\begin{topics}
%	Tema, Subtema.
%\end{topics}


%-----------------------------------------------------------------------------------
% NO BORRAR ESTAS LINEAS!
%-----------------------------------------------------------------------------------
\vspace{0.8cm}
]
%-----------------------------------------------------------------------------------


%===================================================================================

%===================================================================================
% Introducción
%-----------------------------------------------------------------------------------
\section{Introducción}\label{sec:intro}
%-----------------------------------------------------------------------------------
  En esta sección puede incluir una presentación del dominio de su problema,
  los objetivos y motivaciones fundamentales de su investigación así como un
  resumen del estado del arte al respecto.

%===================================================================================



%===================================================================================
% Desarrollo
%-----------------------------------------------------------------------------------
\section{Ejercicios}\label{sec:dev}
%-----------------------------------------------------------------------------------
	Mostramos a continuación los enunciados de los ejercicios propuestos.

%-----------------------------------------------------------------------------------
	\subsection{Ejercicio 1}\label{sub:E1}
%-----------------------------------------------------------------------------------
	Ajuste los siguientes datos a una recta por mínimos cuadrados:
	
	\begin{table}[h]
		\centering
		\begin{tabular}{|c|c|c|c|c|}
			\hline
			x & -1 & 0 & 3 & 7\\ \hline
			F(x) & 2 & 0 & 4 & 7 \\ \hline
		\end{tabular}
		\label{table:EJ1}
	\end{table}

	\begin{enumerate}
		\item {Construya el diagrama de dispersión}
		\item {Estime el valor de la función para x = 1}
	\end{enumerate}

%-----------------------------------------------------------------------------------
	\subsection{Ejercicio 2}\label{sub:OEJ2}
%-----------------------------------------------------------------------------------
	Ajuste los siguientes datos a una recta por mínimos cuadrados:
	
	\begin{table}[h]
		\centering
		\begin{tabular}{|c|c|c|c|c|c|c|}
			\hline
			x & -3 & -1 & 1 & 3 & 5 & 7 \\ \hline
			F(x) & 14 & 4 & 2 & 8 & 22 & 44 \\ \hline
		\end{tabular}
		\label{table:EJ2}
	\end{table}

	\begin{enumerate}
		\item {Construya el diagrama de dispersión}
		\item {Estime el valor de la función para x = 0 y x = 2}
	\end{enumerate}

	\subsection{Ejercicio 3}
	La rapidez de pasos (número de pasos por segundo) es importante para el corredor serio. La rapidez de pasos está estrechamente relacionada con la velocidad y la meta del corredor es alcanzar la óptima rapidez de pasos. Como parte de un estudio, investigadores midieron la rapidez de pasos a 7 diferentes velocidades para 21 de las mejores corredoras del mundo, los valores del promedio de rapidez de pasos para estas mujeres y las velocidades de la prueba se indican en la tabla siguiente.
		
	\begin{table}[h]
		\centering
		\begin{tabular}{|c|c|}
			\hline
			Velocidad (ft/s) & F(x)  \\ \hline
			15.86 & 3.05 \\ \hline
			16.88 & 3.12 \\ \hline
			17.5  & 3.17 \\ \hline
			18.62 & 3.25 \\ \hline
			19.97 & 3.36 \\ \hline
			21.06 & 3.46 \\ \hline
			22.11 &	3.55 \\ \hline
		\end{tabular}
		\label{table:EJ3}
		
	
	\end{table}

	\begin{enumerate}
		\item {	Construya el diagrama de dispersión. ¿Parece ser lineal la relación? ¿Por qué?}
		\item {Encuentre la ecuación de la recta de mejor ajuste.}
		\item {Trace la recta en el diagrama de dispersión.}
		\item {Interprete la pendiente de la ecuación de regresión.}
		\item {Pronostique la rapidez de pasos promedio si la velocidad es de 19 pies por segundo}
		\item {Cuál es la rapidez de pasos si la velocidad es cero. Interprete sus resultados y Explique.}
	\end{enumerate}

%-----------------------------------------------------------------------------------
	\section{Soluciones}\label{sub:answers}
%-----------------------------------------------------------------------------------
		Para dar respuesta a los ejercicios propuestos nos basamos en la búsqueda de la recta de mejor ajuste para los datos mediante el método de los mínimos cuadrados. 
		
		Para poder encontrar la pendiente de la recta de mejor ajuste empleamos la siguiente fórmula computacional:
		
		\begin{center}
			$b_1 = \frac{SS(xy)}{SS(x)}$
			$b_0 = \frac{\sum y - (b_1 * \sum x)}{n}$
		\end{center}
	
		\begin{center}
			$SS(xy) = \sum xy - \frac{\sum x \sum y}{n}$
			$SS(x) = \sum x^2 - \frac{(\sum x)^2}{n}$
		\end{center}
		
		De donde obtenemos la recta:
		
		\begin{center}
			$y = b_0 + b_1x$
		\end{center}
	
		y con ella podemos estimar valores dentro del dominio de nuestros datos.
		
		\subsection{Ejercicios 1 y 2}
			Tras procesar los datos obtuvimos los siguientes gráficos de dispersión junto a su recta de ajuste.
			
			\begin{figure}[h]
				\includegraphics[width=\linewidth]{Images/ejercicio1.jpeg}
			\end{figure}
		
			\begin{figure}[htb]
				\includegraphics[width=\linewidth]{Images/ejercicio2.jpeg}
			\end{figure}
		
			\subsubsection{Ejercicio 1}
			
			Con ecuación de mejor ajuste:
			
			\begin{center}
				$y = 1.522581 + 0.7677419x$
			\end{center}
			
			Para $x = 1$ se estima un valor de $y = 2.29032258064516$.
			
			\subsubsection{Ejercicio 2}
			
			Con ecuación de mejor ajuste:
			
			\begin{center}
				$y = 9.666667 + 3x$
			\end{center}
			
			
			Para $x = 0$ se estima el mismo valor de $b_0$, $y = 9.666667$.
			
			Para $x = 2$ se estima un valor de $y = 15.666667$.
		
		\subsection{Ejercicio 3}
			Comenzamos en este ejercicio mostrando la gráfica de dispersión.
			
			\begin{figure}[htb]
				\includegraphics[width=\linewidth]{Images/ejercicio3norecta.jpeg}
			\end{figure}
		
			De donde nos podemos observar que los datos forman casi una línea recta. Y al evaluar el espectro de los mismos nos percatamos que no se incumple a simple vista ninguno de los requerimientos del modelo lineal. 
			
			Tras los cálculos realizados, llegamos a que la recta que mejor ajusta el modelo es :
			
			\begin{center}
				$y = 1.766077 + 0.08028379x$
			\end{center}
		
			Una de las conclusiones que podemos obtener de la misma es que por cada incremento unitario en la velocidad de las corredoras aumentan sus pasos promedios en 0.08028379 aproximadamente.
			
			Tomando en cuenta nuestro modelo y nuestros datos, para una velocidad de 19 pies por segundo obtenemos un promedio de 3.29146911253888 para su rapidez.
		
			\begin{figure}[htb]
				\includegraphics[width=\linewidth]{Images/ejercicio3.jpeg}
			\end{figure}
		
			Uno de los errores más comunes a la hora de predecir comportamientos sobre el modelo creado radica en la mala utilización del dominio de los datos. Este es el caso que tenemos si le queremos dar una interpretación a la rapidez de pasos con velocidad 0. Aca incurrimos en dos problemas, el primero en que nos salimos del dominio de los datos y segundo que para una velocidad 0, resulta ilógico tener 1.766077 pasos en promedio.
		
					


\label{end}

\end{document}

%===================================================================================
